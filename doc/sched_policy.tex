\section{3-Phase Scheduler}
Traditional batch scheduler just looks at the field of $c_i$ when making scheduling decision,
which just simply ignore the burst buffer node demand from the job.
To be burst-buffer aware,
our scheduling mechanism is divided into 3 phases to adopt to the characteristic of jobs in burst buffer context.
Scheduler schedules jobs in 3 distinct set/queue.
The stage-in set $Q_I$ contains all the jobs that needs to load input data.
The running set $Q_R$ contains all the jobs waiting to be run with loaded data.
The stage-out set $Q_O$ contains all the jobs that needs to write output data to IO nodes.
At anytime, a job can only appear in one of the 3 sets, apparently.
This fact motivates separated scheduling idiom to be used in different phases, or for different job sets/queues.

\subsection{IO-BB Scheduling}
In the stage-in phase, only burst buffer demand is considered.
Scheduling are made based on the value of $bb\_in$ of jobs in $Q_I$.
If we care about data transfer throughput,
we should transfer as much data as possible by doing the following optimization:
\begin{align*}
        & \max \sum_{i\in S} bb\_in_i \textit{   s.t.}\\[1em]
        & \left\{
                \begin{array}{l}
                        S \cup NS = J \\ [1em] \numberthis \label{Equ:MaxTransferData} 
                        \sum_{i\in S} bb\_in_i \leq BB_{available}
                \end{array} 
        \right.
\end{align*}	

If we care about task parallelism, following optimization could help:
\begin{align*}
        & \max |S| \textit{   s.t.} \\[1em]
        & \left\{
                \begin{array}{l}
                        S \cup NS = J \\ [1em] \numberthis \label{Equ:MaxTaskNumber} 
                        \sum_{i\in S} bb\_in_i \leq BB_{available}
                \end{array} 
        \right.
\end{align*}
The number of tasks doing data loading will be maximized.


\subsection{CN-BB Scheduling}
Running jobs require not only compute nodes, but burst buffer to ensure performance and correctness.
Scheduling are accordingly made based on the value of $c$ and $bb\_run$ of jobs in $Q_R$.
To maximize multiple types of resource's utilization,
we convert it to the knapsack problem by defining the value of the $job_i$ as
\begin{equation}
        v_j = \frac{c_i / CN}{rt_i} \times \frac{bb\_run_i / BB}{rt_i}
        \label{Equ:DefValue}
\end{equation}
where $rt_i$ is the running time of $job_i$, the time it takes up the computing nodes.
By definition \ref{Equ:DefValue}, we prefer these tasks that claims to take up node resources with short duration.
Unfortunately, it is difficult to predict $rt_i$ before actually running the job.
Of course we could use the \textit{expected running time} $ert_i$ specified by user.
However, by examining the log traces from ANL, we found that the variance between $rt_i$ and $ert_i$ is significantly different.
For now we can just assume $rt_i$ is constant for all jobs.
In the future, we could adopt machine learning or data mining ideas to predict the running time of a job with demand vector.
Notice that then the value of a task is proportional to $c_i*bb_i$.
The optimizing formula can thus be
\begin{align*}
        & \max \sum_{i \in S}c_i * bb\_run_i \textit{   s.t.}\\[1em]
        & \left\{
                \begin{array}{l}
                        S \cup NS = J \\ [1em]
                        \sum_{i \in S} c_i \leq CN_{available} \\ [1em] \numberthis \label{Equ:MaxProduct} 
                        \sum_{i \in S} bb\_run_i \leq BB_{available}
                \end{array} 
        \right.
\end{align*}


\subsection{BB-IO Scheduling}
Scheduling are made based on the value of $bb\_out$ of jobs in $Q_O$.
Optimization formula for different purpose are almost the same as these in IO-BB scheduling.


\subsection{Solving the Optimization Problems}
It is trivial to show that optimization problem \ref{Equ:MaxTransferData} and \ref{Equ:MaxTaskNumber}
are equivalent to 0-1 knapsack problem.
Problem \ref{Equ:MaxProduct} can be informally treat as two dimension 0-1 knapsack problem.
In fact, we expect all of them are NP-hard problems.
We can solve them with dynamic programming in pseudo polynomial time.
Applying memorization could also help accelerate the solving process.
In fact we are not interested in the optimal result of problem \ref{Equ:MaxTransferData},
\ref{Equ:MaxTaskNumber} and \ref{Equ:MaxProduct} at all but in one combination of jobs
that yields the optimal solution, which can also be easily tracked back down by keeping memorizations.

Since problems \ref{Equ:MaxTransferData},
\ref{Equ:MaxTaskNumber} and \ref{Equ:MaxProduct} are very similar, 
their solution is also highly related.
First, for problem \ref{Equ:MaxTransferData}, the recursive relationship is given by \ref{Equ:MaxTransferDataRecursion}.
In \ref{Equ:MaxTransferDataRecursion}, the memo we keeps during solving is the optimal solution for 
$jobs=(job_1, job_2, \ldots, job_i)$ with $w$ GB of available burst buffer.
It turns out that the recursion for problem \ref{Equ:MaxTaskNumber} is extremely similar to \ref{Equ:MaxTransferDataRecursion}
The memo in \ref{Equ:MaxTaskNumberRecursion} is the same as that in \ref{Equ:MaxTransferDataRecursion}.
The recursion for \ref{Equ:MaxProduct} is a little complicated but still straightforward
Here we should keep the memo of the optimal solution for $jobs=(job_1, job_2, \ldots, job_i)$
with $c$ computing nodes and $w$ GB of burst buffer being available.

Scheduler can obtain an optimal combination of jobs by examining the memo.
Take the problem \ref{Equ:MaxProduct} problem for example.
We start from $dp(n, CN, BB)$.
If $c_n \leq CN$ and $bb\_sin_n \leq BB$, $job_n$ should be scheduled if
$dp(i-1, c, w) \leq dp(i-1, c - c_i, w-bb\_sin_i) + c_i bb\_sin_i$ and recurse with $dp(n-1, CN-c_i, BB-bb\_sin_i)$;
otherwise, $job_n$ should be skipped and we recurse the process on $dp(n-1, CN, BB)$.
The time complexity of solving \ref{Equ:MaxTransferDataRecursion} and \ref{Equ:MaxTaskNumberRecursion} is $O(n\times BB)$.
The time complexity of solving \ref{Equ:MaxProductRecursion} is $O(n\times CN\times BB)$.
Notice that $CN$ and $BB$ may be very large integers, making the pseudo-polynomial algorithm unsuitable
to be used by scheduler.
In practice, we could reduce the time complexity by allocating resource in a coarser granularity.
For example, jobs usually asks for compute node in the unit of 512 nodes;
its demand for burst buffer nodes usually in the unit of 100 GB.
Then we could divide both $CN$ and $c_i$ by 512; divide both $BB$ and $bb_in$ by 100.
It is also possible to reduce the value of $n$, the number of jobs in the queue.
For example, whenever we more resources, we can consider only $\frac{1}{\alpha}n$ jobs in the queue.
This will give us only the partial optimal solution in exchange of less computation complexity.



\begin{strip}
        \begin{align}
                dp(i, w) = & 
                \left\{
                        \begin{array}{l}
                                0, \text{ if $i=0$ } \\ [1em]
                                dp(i-1, w), \text{ if $bb\_in_i > w$} \\ [1em]
                                \max \{ dp(i-1, w), dp(i-1, w-bb\_in_i) + bb\_in_i \}, \text{ if $bb\_in_i \leq w$}
                        \end{array} 
                \right.
                \label{Equ:MaxTransferDataRecursion} 
        \end{align}
\end{strip}

\begin{strip}
        \begin{align}
                dp(i, w) = &
                \left\{
                        \begin{array}{l}
                                0, \text{ if $i=0$ } \\ [1em]
                                dp(i-1, w), \text{ if $bb\_in_i > w$} \\ [1em]
                                \max \{ dp(i-1, w), dp(i-1, w-bb\_in_i) + 1 \}, \text{ if $bb\_in_i \leq w$}
                        \end{array} 
                \right.
                \label{Equ:MaxTaskNumberRecursion}
        \end{align}
\end{strip}


\begin{strip}
        \begin{align}
                dp(i, c, w) = &
                \left\{
                        \begin{array}{l}
                                0, \text{ if $i=0$ } \\ [1em]
                                dp(i-1, c, w), \text{ if $c_i > c$ or $bb\_run_i > w$} \\ [1em]
                                \max \{ dp(i-1, c, w), dp(i-1, c - c_i, w - bb\_run_i) + 1 \}, \text{ if $c_i \leq c$ and $bb\_run_i \leq w$}
                        \end{array} 
                \right.
                \label{Equ:MaxProductRecursion}
        \end{align}
\end{strip}

