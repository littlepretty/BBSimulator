\section{Introduction}

Challenges due to huge amount of scientific data used by applications.
Data requirements for applications at ALCF BG/P.

Storage gap between memory and disk.

Current architecture: IO forwarding nodes/IO gateways with PFSs sit between compute nodes and SAN (storage nodes + disk array)

Burst buffer nodes fills the gap by utilizing various types of memory, for example, non-volatile random-access memory (NVRAM), solid state drive (SSD).

The volume of data read/write may affect the architecture model of burst buffer.

Burst buffer nodes on Trinity is composed of IO nodes and 2 PCIe SSD cards.

In ANL's supercomputing platform, burst buffers are potential distributed on each memory hierarchy.
They may be put at computer nodes, board cabin or IO nodes.
We may also need to use burst buffer as intermediate storage.


Test reference:\cite{Liu:MSST:12}
\cite{SlurmBBGuide}
\cite{Romanus:CORR:15}



