%% bare_conf_compsoc.tex
%% V1.4b
%% 2015/08/26
%% by Michael Shell
%% See:
%% http://www.michaelshell.org/
%% for current contact information.
%%
%% This is a skeleton file demonstrating the use of IEEEtran.cls
%% (requires IEEEtran.cls version 1.8b or later) with an IEEE Computer
%% Society conference paper.
%%
%% Support sites:
%% http://www.michaelshell.org/tex/ieeetran/
%% http://www.ctan.org/pkg/ieeetran
%% and
%% http://www.ieee.org/

%%*************************************************************************
%% Legal Notice:
%% This code is offered as-is without any warranty either expressed or
%% implied; without even the implied warranty of MERCHANTABILITY or
%% FITNESS FOR A PARTICULAR PURPOSE! 
%% User assumes all risk.
%% In no event shall the IEEE or any contributor to this code be liable for
%% any damages or losses, including, but not limited to, incidental,
%% consequential, or any other damages, resulting from the use or misuse
%% of any information contained here.
%%
%% All comments are the opinions of their respective authors and are not
%% necessarily endorsed by the IEEE.
%%
%% This work is distributed under the LaTeX Project Public License (LPPL)
%% ( http://www.latex-project.org/ ) version 1.3, and may be freely used,
%% distributed and modified. A copy of the LPPL, version 1.3, is included
%% in the base LaTeX documentation of all distributions of LaTeX released
%% 2003/12/01 or later.
%% Retain all contribution notices and credits.
%% ** Modified files should be clearly indicated as such, including  **
%% ** renaming them and changing author support contact information. **
%%*************************************************************************


% *** Authors should verify (and, if needed, correct) their LaTeX system  ***
% *** with the testflow diagnostic prior to trusting their LaTeX platform ***
% *** with production work. The IEEE's font choices and paper sizes can   ***
% *** trigger bugs that do not appear when using other class files.       ***                          ***
% The testflow support page is at:
% http://www.michaelshell.org/tex/testflow/



\documentclass[conference,compsoc]{IEEEtran}
% Some/most Computer Society conferences require the compsoc mode option,
% but others may want the standard conference format.
%
% If IEEEtran.cls has not been installed into the LaTeX system files,
% manually specify the path to it like:
% \documentclass[conference,compsoc]{../sty/IEEEtran}





% Some very useful LaTeX packages include:
% (uncomment the ones you want to load)


% *** MISC UTILITY PACKAGES ***
%
\usepackage{ifpdf}
% Heiko Oberdiek's ifpdf.sty is very useful if you need conditional
% compilation based on whether the output is pdf or dvi.
% usage:
% \ifpdf
%   % pdf code
% \else
%   % dvi code
% \fi
% The latest version of ifpdf.sty can be obtained from:
% http://www.ctan.org/pkg/ifpdf
% Also, note that IEEEtran.cls V1.7 and later provides a builtin
% \ifCLASSINFOpdf conditional that works the same way.
% When switching from latex to pdflatex and vice-versa, the compiler may
% have to be run twice to clear warning/error messages.






% *** CITATION PACKAGES ***
%
\ifCLASSOPTIONcompsoc
  % IEEE Computer Society needs nocompress option
  % requires cite.sty v4.0 or later (November 2003)
  \usepackage[nocompress]{cite}
\else
  % normal IEEE
  \usepackage{cite}
\fi
% cite.sty was written by Donald Arseneau
% V1.6 and later of IEEEtran pre-defines the format of the cite.sty package
% \cite{} output to follow that of the IEEE. Loading the cite package will
% result in citation numbers being automatically sorted and properly
% "compressed/ranged". e.g., [1], [9], [2], [7], [5], [6] without using
% cite.sty will become [1], [2], [5]--[7], [9] using cite.sty. cite.sty's
% \cite will automatically add leading space, if needed. Use cite.sty's
% noadjust option (cite.sty V3.8 and later) if you want to turn this off
% such as if a citation ever needs to be enclosed in parenthesis.
% cite.sty is already installed on most LaTeX systems. Be sure and use
% version 5.0 (2009-03-20) and later if using hyperref.sty.
% The latest version can be obtained at:
% http://www.ctan.org/pkg/cite
% The documentation is contained in the cite.sty file itself.
%
% Note that some packages require special options to format as the Computer
% Society requires. In particular, Computer Society  papers do not use
% compressed citation ranges as is done in typical IEEE papers
% (e.g., [1]-[4]). Instead, they list every citation separately in order
% (e.g., [1], [2], [3], [4]). To get the latter we need to load the cite
% package with the nocompress option which is supported by cite.sty v4.0
% and later.





% *** GRAPHICS RELATED PACKAGES ***
%
\ifCLASSINFOpdf
  % \usepackage[pdftex]{graphicx}
  % declare the path(s) where your graphic files are
  % \graphicspath{{../pdf/}{../jpeg/}}
  % and their extensions so you won't have to specify these with
  % every instance of \includegraphics
  % \DeclareGraphicsExtensions{.pdf,.jpeg,.png}
\else
  % or other class option (dvipsone, dvipdf, if not using dvips). graphicx
  % will default to the driver specified in the system graphics.cfg if no
  % driver is specified.
  % \usepackage[dvips]{graphicx}
  % declare the path(s) where your graphic files are
  % \graphicspath{{../eps/}}
  % and their extensions so you won't have to specify these with
  % every instance of \includegraphics
  % \DeclareGraphicsExtensions{.eps}
\fi
% graphicx was written by David Carlisle and Sebastian Rahtz. It is
% required if you want graphics, photos, etc. graphicx.sty is already
% installed on most LaTeX systems. The latest version and documentation
% can be obtained at: 
% http://www.ctan.org/pkg/graphicx
% Another good source of documentation is "Using Imported Graphics in
% LaTeX2e" by Keith Reckdahl which can be found at:
% http://www.ctan.org/pkg/epslatex
%
% latex, and pdflatex in dvi mode, support graphics in encapsulated
% postscript (.eps) format. pdflatex in pdf mode supports graphics
% in .pdf, .jpeg, .png and .mps (metapost) formats. Users should ensure
% that all non-photo figures use a vector format (.eps, .pdf, .mps) and
% not a bitmapped formats (.jpeg, .png). The IEEE frowns on bitmapped formats
% which can result in "jaggedy"/blurry rendering of lines and letters as
% well as large increases in file sizes.
%
% You can find documentation about the pdfTeX application at:
% http://www.tug.org/applications/pdftex





% *** MATH PACKAGES ***
%
\usepackage{amsmath}
\usepackage{cuted}
% A popular package from the American Mathematical Society that provides
% many useful and powerful commands for dealing with mathematics.
%
% Note that the amsmath package sets \interdisplaylinepenalty to 10000
% thus preventing page breaks from occurring within multiline equations. Use:
%\interdisplaylinepenalty=2500
% after loading amsmath to restore such page breaks as IEEEtran.cls normally
% does. amsmath.sty is already installed on most LaTeX systems. The latest
% version and documentation can be obtained at:
% http://www.ctan.org/pkg/amsmath
\newcommand\numberthis{\addtocounter{equation}{1}\tag{\theequation}}




% *** SPECIALIZED LIST PACKAGES ***
%
\usepackage{algorithmic}
% algorithmic.sty was written by Peter Williams and Rogerio Brito.
% This package provides an algorithmic environment fo describing algorithms.
% You can use the algorithmic environment in-text or within a figure
% environment to provide for a floating algorithm. Do NOT use the algorithm
% floating environment provided by algorithm.sty (by the same authors) or
% algorithm2e.sty (by Christophe Fiorio) as the IEEE does not use dedicated
% algorithm float types and packages that provide these will not provide
% correct IEEE style captions. The latest version and documentation of
% algorithmic.sty can be obtained at:
% http://www.ctan.org/pkg/algorithms
% Also of interest may be the (relatively newer and more customizable)
% algorithmicx.sty package by Szasz Janos:
% http://www.ctan.org/pkg/algorithmicx




% *** ALIGNMENT PACKAGES ***
%
\usepackage{array}
% Frank Mittelbach's and David Carlisle's array.sty patches and improves
% the standard LaTeX2e array and tabular environments to provide better
% appearance and additional user controls. As the default LaTeX2e table
% generation code is lacking to the point of almost being broken with
% respect to the quality of the end results, all users are strongly
% advised to use an enhanced (at the very least that provided by array.sty)
% set of table tools. array.sty is already installed on most systems. The
% latest version and documentation can be obtained at:
% http://www.ctan.org/pkg/array


% IEEEtran contains the IEEEeqnarray family of commands that can be used to
% generate multiline equations as well as matrices, tables, etc., of high
% quality.




% *** SUBFIGURE PACKAGES ***
%\ifCLASSOPTIONcompsoc
%  \usepackage[caption=false,font=footnotesize,labelfont=sf,textfont=sf]{subfig}
%\else
%  \usepackage[caption=false,font=footnotesize]{subfig}
%\fi
% subfig.sty, written by Steven Douglas Cochran, is the modern replacement
% for subfigure.sty, the latter of which is no longer maintained and is
% incompatible with some LaTeX packages including fixltx2e. However,
% subfig.sty requires and automatically loads Axel Sommerfeldt's caption.sty
% which will override IEEEtran.cls' handling of captions and this will result
% in non-IEEE style figure/table captions. To prevent this problem, be sure
% and invoke subfig.sty's "caption=false" package option (available since
% subfig.sty version 1.3, 2005/06/28) as this is will preserve IEEEtran.cls
% handling of captions.
% Note that the Computer Society format requires a sans serif font rather
% than the serif font used in traditional IEEE formatting and thus the need
% to invoke different subfig.sty package options depending on whether
% compsoc mode has been enabled.
%
% The latest version and documentation of subfig.sty can be obtained at:
% http://www.ctan.org/pkg/subfig




% *** FLOAT PACKAGES ***
%
%\usepackage{fixltx2e}
% fixltx2e, the successor to the earlier fix2col.sty, was written by
% Frank Mittelbach and David Carlisle. This package corrects a few problems
% in the LaTeX2e kernel, the most notable of which is that in current
% LaTeX2e releases, the ordering of single and double column floats is not
% guaranteed to be preserved. Thus, an unpatched LaTeX2e can allow a
% single column figure to be placed prior to an earlier double column
% figure.
% Be aware that LaTeX2e kernels dated 2015 and later have fixltx2e.sty's
% corrections already built into the system in which case a warning will
% be issued if an attempt is made to load fixltx2e.sty as it is no longer
% needed.
% The latest version and documentation can be found at:
% http://www.ctan.org/pkg/fixltx2e


%\usepackage{stfloats}
% stfloats.sty was written by Sigitas Tolusis. This package gives LaTeX2e
% the ability to do double column floats at the bottom of the page as well
% as the top. (e.g., "\begin{figure*}[!b]" is not normally possible in
% LaTeX2e). It also provides a command:
%\fnbelowfloat
% to enable the placement of footnotes below bottom floats (the standard
% LaTeX2e kernel puts them above bottom floats). This is an invasive package
% which rewrites many portions of the LaTeX2e float routines. It may not work
% with other packages that modify the LaTeX2e float routines. The latest
% version and documentation can be obtained at:
% http://www.ctan.org/pkg/stfloats
% Do not use the stfloats baselinefloat ability as the IEEE does not allow
% \baselineskip to stretch. Authors submitting work to the IEEE should note
% that the IEEE rarely uses double column equations and that authors should try
% to avoid such use. Do not be tempted to use the cuted.sty or midfloat.sty
% packages (also by Sigitas Tolusis) as the IEEE does not format its papers in
% such ways.
% Do not attempt to use stfloats with fixltx2e as they are incompatible.
% Instead, use Morten Hogholm'a dblfloatfix which combines the features
% of both fixltx2e and stfloats:
%
% \usepackage{dblfloatfix}
% The latest version can be found at:
% http://www.ctan.org/pkg/dblfloatfix




% *** PDF, URL AND HYPERLINK PACKAGES ***
%
\usepackage{url}
% url.sty was written by Donald Arseneau. It provides better support for
% handling and breaking URLs. url.sty is already installed on most LaTeX
% systems. The latest version and documentation can be obtained at:
% http://www.ctan.org/pkg/url
% Basically, \url{my_url_here}.




% *** Do not adjust lengths that control margins, column widths, etc. ***
% *** Do not use packages that alter fonts (such as pslatex).         ***
% There should be no need to do such things with IEEEtran.cls V1.6 and later.
% (Unless specifically asked to do so by the journal or conference you plan
% to submit to, of course. )


% correct bad hyphenation here
\hyphenation{op-tical net-works semi-conduc-tor}


\begin{document}
%
% paper title
% Titles are generally capitalized except for words such as a, an, and, as,
% at, but, by, for, in, nor, of, on, or, the, to and up, which are usually
% not capitalized unless they are the first or last word of the title.
% Linebreaks \\ can be used within to get better formatting as desired.
% Do not put math or special symbols in the title.
\title{A Parallelism Oriented Three Phase Based Burst Buffering Aware Scheduler}


% author names and affiliations
% use a multiple column layout for up to three different
% affiliations
\author{\IEEEauthorblockN{Michael Shell}
\IEEEauthorblockA{School of Electrical and\\Computer Engineering\\
Georgia Institute of Technology\\
Atlanta, Georgia 30332--0250\\
Email: http://www.michaelshell.org/contact.html}
\and
\IEEEauthorblockN{Homer Simpson}
\IEEEauthorblockA{Twentieth Century Fox\\
Springfield, USA\\
Email: homer@thesimpsons.com}
\and
\IEEEauthorblockN{James Kirk\\ and Montgomery Scott}
\IEEEauthorblockA{Starfleet Academy\\
San Francisco, California 96678-2391\\
Telephone: (800) 555--1212\\
Fax: (888) 555--1212}}

% conference papers do not typically use \thanks and this command
% is locked out in conference mode. If really needed, such as for
% the acknowledgment of grants, issue a \IEEEoverridecommandlockouts
% after \documentclass

% for over three affiliations, or if they all won't fit within the width
% of the page (and note that there is less available width in this regard for
% compsoc conferences compared to traditional conferences), use this
% alternative format:
% 
%\author{\IEEEauthorblockN{Michael Shell\IEEEauthorrefmark{1},
%Homer Simpson\IEEEauthorrefmark{2},
%James Kirk\IEEEauthorrefmark{3}, 
%Montgomery Scott\IEEEauthorrefmark{3} and
%Eldon Tyrell\IEEEauthorrefmark{4}}
%\IEEEauthorblockA{\IEEEauthorrefmark{1}School of Electrical and Computer Engineering\\
%Georgia Institute of Technology,
%Atlanta, Georgia 30332--0250\\ Email: see http://www.michaelshell.org/contact.html}
%\IEEEauthorblockA{\IEEEauthorrefmark{2}Twentieth Century Fox, Springfield, USA\\
%Email: homer@thesimpsons.com}
%\IEEEauthorblockA{\IEEEauthorrefmark{3}Starfleet Academy, San Francisco, California 96678-2391\\
%Telephone: (800) 555--1212, Fax: (888) 555--1212}
%\IEEEauthorblockA{\IEEEauthorrefmark{4}Tyrell Inc., 123 Replicant Street, Los Angeles, California 90210--4321}}




% use for special paper notices
%\IEEEspecialpapernotice{(Invited Paper)}




% make the title area
\maketitle

% As a general rule, do not put math, special symbols or citations
% in the abstract
\begin{abstract}
In an computing world full of Big Data, moderate I/O performance will drastically slow down the overall execution time of applications.
Burst buffer nodes comes to rescue by providing both higher bandwidth and IOPS.
In this paper, we model the execution of applications on supercomputing system equipped with burst buffer nodes.
Computing nodes are able to be freed earlier due to these efficient and reliable IO broker, 
possibly improving the utilization of task execution pipeline.
We thus characterize generic applications by 3 phases: \textit{data stage in} phase, \textit{running} phase and \textit{data stage out} phase.
Resources in these phases are intuitively scheduled separately targeting on possible different goals.
In both data stage input and output phase, it is possible to maximize data transfer throughput or task parallelism by allocating burst buffer.
In the running phase, we consider maximize the value of scheduling tasks with both computing node resources and burst buffer node demand.
Further investigating shows that all aforementioned optimization problems are equivalent to 0-1 knapsack problem,
requiring exponential time to solve.
We use dynamic programming and memorization technique to give precise solutions.
We simulate scheduling system in the scenarios both with and without burst buffer on a discrete event simulator to demonstrate that
1) burst buffer node can accelerate the execution pipeline of all task, thus improving the utilization of the computing nodes as well as task response time
2) our scheduling algorithm can further benefit burst buffer equipped system.
\end{abstract}

% no keywords




% For peer review papers, you can put extra information on the cover
% page as needed:
% \ifCLASSOPTIONpeerreview
% \begin{center} \bfseries EDICS Category: 3-BBND \end{center}
% \fi
%
% For peerreview papers, this IEEEtran command inserts a page break and
% creates the second title. It will be ignored for other modes.
\IEEEpeerreviewmaketitle



\section{Introduction}
% no \IEEEPARstart
This demo file is intended to serve as a ``starter file''
for IEEE Computer Society conference papers produced under \LaTeX\ using
IEEEtran.cls version 1.8b and later.
% You must have at least 2 lines in the paragraph with the drop letter
% (should never be an issue)
I wish you the best of success.

\hfill mds
 
\hfill August 26, 2015

\section{Model for Computing System}
\subsection{System Resources}
The system contains resources like compute node, main memory, burst buffer and IO node.
In most system architecture, compute node are coupled with main memmory.
We assume IO nodes are always available.
Therefore, the schedule targets are modeled as compute node and burst buffer.
We consider a system with $CN$ compute nodes and $BB$ GB of burst buffer.
Burst buffer are able to be shared by applications with any fraction of total amount.

\subsection{User Jobs}
Users request the system to run their applications as jobs.
We define $J = (job_1, job_2,..., job_n)$ to be the set of all jobs.
Job consists of 3 different phases.
In the first phases, input data are load in from IO nodes to burst buffer.
We refer this phase as \textit{stage-in} phase.
Then the job runs on the compute nodes;
except reading data from burst buffer, there may or may not be any interaction between computer node and burst buffer.
This phase is called \textit{running} phase.
When computation is done, output data needs to be staged out to IO nodes from burst buffer, namely \textit{stage-out} phase.
Burst buffer plays as IO agent for compute node.
From the view point of compute node, it uses burst buffer as IO nodes.
For a job at the stage-in phase, compute nodes is not allocated to it yet.
When scheduler allocate compute nodes to a job, these nodes are exclusively used by this job until it enters stage-out phase.
Compute nodes are release as soon as computation is done, meanning they are available for jobs waiting to be run.
Staging output out to IO node is the bussiness just between burst buffer and IO nodes.


\subsection{Resource Demand}
User typically provides its resource demand for her job.
Therefore, each job is associated with a demand vector in the form of $(c, bb\_sin, bb)$,
where $c$ is the number of needed compute node in running phase,
$bb\_sin$ is the volumn of burst buffer user predicted for file stage in,
$bb$ is the volumn of burst buffer user preferred.
A benign user should set $bb = \max\{bb\_running, bb\_sout\}$.
However, we do not make any assumption about $bb\_running$ and $bb\_sout$ because it is nontrivial to predict them,
both for system scheduler and application owner.

\subsection{3-Phase Scheduler}
Scheduling is divided into 3 phases to adopt to the charateristic of jobs.
Scheduler schedules jobs in 3 distince set.
The stage-in set $Q_I$ contains all the jobs that needs to load input data.
The running set $Q_R$ contains all the jobs waiting to be run with loaded data.
The stage-out set $Q_O$ contains all the jobs that needs to write output data to IO nodes.
At anytime, a job can only appear in one of the 3 sets.
This fact motivates seperated scheduling idiom.

\subsubsection*{IO-BB Scheduling}
Scheduling are made based on the value of $bb_{sin}$ of jobs in $Q_I$.
If we care about data transfer throughput, 
we should transfer as much data as possible by doing the following optimization:
\begin{align*}
        & \max \sum_{i\in S} bb\_sin_i \textit{   s.t.}\\[1em]
        & \left\{
                \begin{array}{l}
                        S \cup NS = J \\ [1em] \numberthis \label{Equ:MaxTransferData} 
                        \sum_{i\in S} bb\_sin_i \leq BB
                \end{array} 
        \right.
\end{align*}	

If we care about task parallelism, following optimization could help:
\begin{align*}
        & \max |S| \textit{   s.t.} \\[1em]
        & \left\{
                \begin{array}{l}
                        S \cup NS = J \\ [1em] \numberthis \label{Equ:MaxTaskNumber} 
                        \sum_{i\in S} bb\_sin_i \leq BB
                \end{array} 
        \right.
\end{align*}
The number of tasks doing data loading will be maximized.


\subsubsection*{CN-BB Scheduling}
Scheduling are made based on the value of $c$ and $bb$ of jobs in $Q_R$.
To maximize the resource utilization, we define the value of $job_i$ as
\begin{equation}
        v_j = \frac{c_i / CN}{rt_i} \times \frac{bb_i / BB}{rt_i}
        \label{Equ:DefValue}
\end{equation}
where $rt_i$ is the running time of $job_i$, the time it takes up the computing nodes.
By definition \ref{Equ:DefValue}, we prefer these tasks that claims to take up node resources with short duration.
Unfortunately, it is difficult to predict $rt_i$ before actually running the job.
For now we can just assume $rt_i$ is constant for all jobs.
In the future, we could adopt machine learning or data mining ideas to predict the running time of a job with demand vector.

Notice that the value of a task is proportional to $c_i*bb_i$.
The optimizing formula can thus be
\begin{align*}
        & \max \sum_{i \in S}c_i * bb_i \textit{   s.t.}\\[1em]
        & \left\{
                \begin{array}{l}
                        S \cup NS = J \\ [1em]
                        \sum_{i \in S}c_i \leq CN \\ [1em] \numberthis \label{Equ:MaxProduct} 
                        \sum_{i\in S} bb_i \leq BB
                \end{array} 
        \right.
\end{align*}	


\subsubsection*{BB-IO Scheduling}
Scheduling are made based on the value of $bb$ of jobs in $Q_O$.
Optimization formula for different purpose are almost the same to these in IO-BB scheduling.


\section{Solving the Optimization Problems}
It is trivial to show that optimization problem \ref{Equ:MaxTransferData} and \ref{Equ:MaxTaskNumber}
are equivalent to 0-1 knapsack problem.
Problem \ref{Equ:MaxProduct} can be informally treat as two dimension 0-1 knapsack problem.
In fact, we expect all of them are NP-hard problems.
We can solve them with dynamic programming in pseudo polynomial time.
Memorization is also applied to accelerate the solving process.
In fact we are not interested in the optimal solution at all but in one combination of jobs
that yields the optimal solution, which can also be easily tracked back down by keeping memorizations.

\begin{strip}
        \begin{align}
                dp(i, w) = & 
                \left\{
                        \begin{array}{l}
                                0, \text{ if $i=0$ } \\ [1em]
                                dp(i-1, w), \text{ if $bb\_sin_i > w$} \\ [1em]
                                \max \{ dp(i-1, w), dp(i-1, w-bb\_sin_i) + bb\_sin_i \}, \text{ if $bb\_sin_i \leq w$}
                        \end{array} 
                \right.
                \label{Equ:MaxTransferDataRecursion} 
                \\[1em]
                dp(i, w) = &
                \left\{
                        \begin{array}{l}
                                0, \text{ if $i=0$ } \\ [1em]
                                dp(i-1, w), \text{ if $bb\_sin_i > w$} \\ [1em]
                                \max \{ dp(i-1, w), dp(i-1, w-bb\_sin_i) + 1 \}, \text{ if $bb\_sin_i \leq w$}
                        \end{array} 
                \right.
                \label{Equ:MaxTaskNumberRecursion}
                \\[1em]
                dp(i, c, w) = &
                \left\{
                        \begin{array}{l}
                                0, \text{ if $i=0$ } \\ [1em]
                                dp(i-1, c, w), \text{ if $c_i > c$ or $bb\_sin_i > w$} \\ [1em]
                                \max \{ dp(i-1, c, w), dp(i-1, c - c_i, w-bb\_sin_i) + 1 \}, \text{ if $c_i \leq c$ and $bb\_sin_i \leq w$}
                        \end{array} 
                \right.
                \label{Equ:MaxProductRecursion}
        \end{align}
\end{strip}

Since problems \ref{Equ:MaxTransferData} -- \ref{Equ:MaxProduct} are very similar, 
their solution is also highly related.
First, for problem \ref{Equ:MaxTransferData}, the recursive relationship is given by \ref{Equ:MaxTransferDataRecursion}.
In \ref{Equ:MaxTransferDataRecursion}, the memo we keeps during solving is the optimal solution for 
$jobs=(job_1, job_2, \ldots, job_i)$ with $w$ GB of burst buffer.
It turns out that the recursion for problem \ref{Equ:MaxTaskNumber} is extremely similar to \ref{Equ:MaxTransferDataRecursion}
The memo in \ref{Equ:MaxTaskNumberRecursion} is the same as that in \ref{Equ:MaxTransferDataRecursion}.
The recursion for \ref{Equ:MaxProduct} is a little complicated but still straightforward
Here we should keep the memo of the optimal solution for $jobs=(job_1, job_2, \ldots, job_i)$
with $c$ computing nodes and $w$ GB of burst buffer.

Scheduler can obtain an optimal combination of jobs by examining the memo.
Take the problem \ref{Equ:MaxProduct} problem for example.
We start from $dp(n, CN, BB)$; if $c_n \leq CN$ and $bb\_sin_n \leq BB$, $job_n$ should be scheduled if
$dp(i-1, c, w) \leq dp(i-1, c - c_i, w-bb\_sin_i) + c_i bb\_sin_i$ and recurse with $dp(n-1, CN-c_i, BB-bb\_sin_i$;
otherwise, $job_n$ should be skipped and we recurse the process on $dp(n-1, CN, BB)$.
The time complexity of solving \ref{Equ:MaxTransferDataRecursion} and \ref{Equ:MaxTaskNumberRecursion} is $O(n\times BB)$.
The time complexity of solving \ref{Equ:MaxProductRecursion} is $O(n\times CN\times BB)$.



% An example of a floating figure using the graphicx package.
% Note that \label must occur AFTER (or within) \caption.
% For figures, \caption should occur after the \includegraphics.
% Note that IEEEtran v1.7 and later has special internal code that
% is designed to preserve the operation of \label within \caption
% even when the captionsoff option is in effect. However, because
% of issues like this, it may be the safest practice to put all your
% \label just after \caption rather than within \caption{}.
%
% Reminder: the "draftcls" or "draftclsnofoot", not "draft", class
% option should be used if it is desired that the figures are to be
% displayed while in draft mode.
%
%\begin{figure}[!t]
%\centering
%\includegraphics[width=2.5in]{myfigure}
% where an .eps filename suffix will be assumed under latex, 
% and a .pdf suffix will be assumed for pdflatex; or what has been declared
% via \DeclareGraphicsExtensions.
%\caption{Simulation results for the network.}
%\label{fig_sim}
%\end{figure}

% Note that the IEEE typically puts floats only at the top, even when this
% results in a large percentage of a column being occupied by floats.


% An example of a double column floating figure using two subfigures.
% (The subfig.sty package must be loaded for this to work.)
% The subfigure \label commands are set within each subfloat command,
% and the \label for the overall figure must come after \caption.
% \hfil is used as a separator to get equal spacing.
% Watch out that the combined width of all the subfigures on a 
% line do not exceed the text width or a line break will occur.
%
%\begin{figure*}[!t]
%\centering
%\subfloat[Case I]{\includegraphics[width=2.5in]{box}%
%\label{fig_first_case}}
%\hfil
%\subfloat[Case II]{\includegraphics[width=2.5in]{box}%
%\label{fig_second_case}}
%\caption{Simulation results for the network.}
%\label{fig_sim}
%\end{figure*}
%
% Note that often IEEE papers with subfigures do not employ subfigure
% captions (using the optional argument to \subfloat[]), but instead will
% reference/describe all of them (a), (b), etc., within the main caption.
% Be aware that for subfig.sty to generate the (a), (b), etc., subfigure
% labels, the optional argument to \subfloat must be present. If a
% subcaption is not desired, just leave its contents blank,
% e.g., \subfloat[].


% An example of a floating table. Note that, for IEEE style tables, the
% \caption command should come BEFORE the table and, given that table
% captions serve much like titles, are usually capitalized except for words
% such as a, an, and, as, at, but, by, for, in, nor, of, on, or, the, to
% and up, which are usually not capitalized unless they are the first or
% last word of the caption. Table text will default to \footnotesize as
% the IEEE normally uses this smaller font for tables.
% The \label must come after \caption as always.
%
%\begin{table}[!t]
%% increase table row spacing, adjust to taste
%\renewcommand{\arraystretch}{1.3}
% if using array.sty, it might be a good idea to tweak the value of
% \extrarowheight as needed to properly center the text within the cells
%\caption{An Example of a Table}
%\label{table_example}
%\centering
%% Some packages, such as MDW tools, offer better commands for making tables
%% than the plain LaTeX2e tabular which is used here.
%\begin{tabular}{|c||c|}
%\hline
%One & Two\\
%\hline
%Three & Four\\
%\hline
%\end{tabular}
%\end{table}


% Note that the IEEE does not put floats in the very first column
% - or typically anywhere on the first page for that matter. Also,
% in-text middle ("here") positioning is typically not used, but it
% is allowed and encouraged for Computer Society conferences (but
% not Computer Society journals). Most IEEE journals/conferences use
% top floats exclusively. 
% Note that, LaTeX2e, unlike IEEE journals/conferences, places
% footnotes above bottom floats. This can be corrected via the
% \fnbelowfloat command of the stfloats package.




\section{Conclusion}
The conclusion goes here.




% conference papers do not normally have an appendix



% use section* for acknowledgment
\ifCLASSOPTIONcompsoc
  % The Computer Society usually uses the plural form
  \section*{Acknowledgments}
\else
  % regular IEEE prefers the singular form
  \section*{Acknowledgment}
\fi


The authors would like to thank...





% trigger a \newpage just before the given reference
% number - used to balance the columns on the last page
% adjust value as needed - may need to be readjusted if
% the document is modified later
%\IEEEtriggeratref{8}
% The "triggered" command can be changed if desired:
%\IEEEtriggercmd{\enlargethispage{-5in}}

% references section

% can use a bibliography generated by BibTeX as a .bbl file
% BibTeX documentation can be easily obtained at:
% http://mirror.ctan.org/biblio/bibtex/contrib/doc/
% The IEEEtran BibTeX style support page is at:
% http://www.michaelshell.org/tex/ieeetran/bibtex/
%\bibliographystyle{IEEEtran}
% argument is your BibTeX string definitions and bibliography database(s)
%\bibliography{IEEEabrv,../bib/paper}
%
% <OR> manually copy in the resultant .bbl file
% set second argument of \begin to the number of references
% (used to reserve space for the reference number labels box)
\begin{thebibliography}{1}

\bibitem{IEEEhowto:kopka}
H.~Kopka and P.~W. Daly, \emph{A Guide to \LaTeX}, 3rd~ed.\hskip 1em plus
  0.5em minus 0.4em\relax Harlow, England: Addison-Wesley, 1999.

\end{thebibliography}




% that's all folks
\end{document}


