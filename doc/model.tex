\section{System Modeling and Problem Formulation}
\label{Sec:Model}

\begin{table}[!t] 
        \renewcommand{\arraystretch}{1.3}
        \caption{Summary of Symbols}
        \label{Tab:Symbols}
        \centering
        \begin{tabular}{l|l}
                \hline
                $CN$ & number of total compute nodes \\
                $BB$ & amount of total burst buffer nodes \\
                $CN_{available}$ & number of available compute nodes \\
                $BB_{available}$ & amount of available burst buffer nodes \\
                $c_i$ & compute node demand of $job_i$ \\
                $bb\_in_i$ & burst buffer demand of $job_i$ at stage-in phase \\
                $bb\_run_i$ & burst buffer demand of $job_i$ at running phase \\
                $bb\_out_i$ & burst buffer demand of $job_i$ at stage-out phase \\
                $rt_i$ & running time of $job_i$ \\
                $ert_i$ & estimated or expected running time of $job_i$ \\
                $Q_I, Q_R, Q_O$ & queues for stage-in, running, stage-out phase \\
                $J_{Q_I}, J_{Q_R}, J_{Q_O}$ & set of jobs in corresponding queue \\
                $S$ & set of jobs selected by scheduler \\
                $v_i$ & value of $job_i \in J_{Q_R}$ defined by Equation~\ref{Equ:DefValue}\\
                $dp(\cdot)$ & memo used in dynamic programming \\
                \hline
        \end{tabular}
\end{table}

\subsection{System Resources}
The system contains resources like compute node, main memory, burst buffer and IO node.
In most system architecture, compute nodes are coupled with main memory.
Since IO capacity is more than sufficient, We assume IO nodes are always available.
Therefore, the schedule targets are modeled as compute node and burst buffer.
We consider a system with $CN$ compute nodes and $BB$ GB of burst buffer.
Burst buffer are able to be shared by applications with any fraction of total amount.
Later we will see that when doing optimal scheduling, we may need to allocate by a fixed
fraction of the total resources(number of compute nodes or amount of burst buffer).

\subsection{Three-Phase Model for HPC Jobs}
Users request the system to run their applications as jobs.
We define $J = (job_1, job_2,..., job_n)$ to be the set of all jobs in the system during
a period of time interested.
Job consists of 3 different phases.
In the first phases, input data are load in from IO nodes to burst buffer.
We refer this phase as \textit{stage-in} phase.
It is very easy for user to provide the amount of load-in data because input data is available before application execution. 
Then the job can run on the compute nodes when been scheduled;
except reading data from burst buffer, there may or may not be any interaction between computer node and burst buffer.
These interactions are mainly due to fault tolerance reasons, like check-pointing.
This phase is called \textit{running} phase.
When computation is done, output data needs to be staged out to IO nodes from memory or burst buffer, namely \textit{stage-out} phase.
Burst buffer plays as IO agent for compute node.
From the view point of compute node, it uses burst buffer as IO nodes.
For a job at the stage-in phase, compute nodes is not allocated to it yet but it will not effect reading in data.
When scheduler allocate compute nodes to a job, these nodes are exclusively used by this job until it enters stage-out phase.
Compute nodes are release as soon as computation is done, meaning they are available for jobs waiting to be run.
Staging output data out to IO node is the business just between burst buffer and IO nodes.


\subsection{Three-Phase Resource Demand}
Users typically provide their resource demand for their jobs.
This is the most important information scheduler could get from the users.
Therefore, each job is associated with a demand vector in the form of $(c, rt)$
where $c$ is the number of needed compute node in running phase,
$rt$ is the requested runtime user can provide to help scheduler make decisions.
To be burst buffer aware, demand vector should be augmented
with fields defining user's request for burst buffer capacity.
We consider two possible ways to do so.
Corresponding to the typical usage cases of burst buffer,
user may provide a \textit{burst buffer triple} $(bb\_in, bb\_run, bb\_out)$,
where $bb\_in$ is the volume of burst buffer user predicted for staging in data files,
$bb\_run$ is the volume of burst buffer user preferred for checkpointing during running,
$bb\_out$ is the volume of burst buffer needed to hold the resulting output data.
A job that providing demand vector augmented with
burst buffer triple is called a \textit{3-phase modeled job}.
Alternatively, a lazy user may just use the $\max\{bb\_in, bb\_run, bb\_out\}$ at every stage.
We do not make any assumption about $bb\_run$ and $bb\_out$ because it is nontrivial to predict them,
both for system scheduler and application owner.
We refer this augmentation as
\textit{1-phase modeled jobs} thereafter.


