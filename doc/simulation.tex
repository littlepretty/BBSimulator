\section{Event-driven Burst Buffer Simulator}
\label{Sec:Simulation}

We develop an event-driven simulator for burst buffer enabled HPC system, named \textbf{BBSim},
from scratch in Python to mimic Cerberus scheduling Trinity.
It roughly consists of 1,400 lines of code.
Figure~\ref{Fig:JobFSM} illustrates the simulation lifetime of a job in BBSim
using extended finite state machine (EFSM).
The \textit{state} of the job changes depending on current status and
happening \textit{events}, along with which an \textit{action} is taken.
For example, at the very beginning, submitted $job_i$ will enter \textbf{Waiting Stage-in} state;
at the meanwhile, $job_i$ is enqueued to $Q_I$ and scheduler is triggered to do scheduling.

Whenever job enters one of its 3 phases, system resources are allocated:
\begin{itemize}
        \item $BB_{in}$ TB amount of burst buffer are allocated upon entering stage-in phase.
        \item $CN$ number of compute nodes and $BB_{run}$ amount of burst buffer
                are allocated upon entering running phase.
        \item $BB_{out}$ TB of burst buffer are allocated upon entering stage-out phase.
\end{itemize}
Various \texttt{release()} actions are of importance because, in addition to submission,
\texttt{schedule()} is invoked to schedule waiting jobs
whenever system resources are released:
\begin{itemize}
        \item When job's data is loaded from burst buffer to memory,
                burst buffer allocated at stage-in phase are released.
        \item When job finishes running, system reclaims burst buffer nodes
                used for checkpointing.
        \item When job's data is written to burst buffer from memory,
                compute nodes taken by job are returned.
        \item When job's data is staged out to external disk,
                its burst buffer nodes are released.
\end{itemize}

Notice that resource can only be released when certain phase is finished.
Therefore, any \textit{dispatch} event, caused by \texttt{schedule()} action, actually happens
at the meanwhile of a certain \textit{finish} event.
The timestamps of all possible \textit{finish} events are calculated by following model:
\begin{align}
        & TS_{f\_stagein} = TS_{s\_stagein} + \frac{bb\_in}{BW_{io\_to\_bb}}\label{Equ:FinIn} \\
        & TS_{f\_loadin} = TS_{s\_run} + \frac{bb\_in}{BW_{bb\_to\_cn}}\label{Equ:FinMemIn} \\
        & TS_{f\_run} = TS_{f\_loadin} + \frac{bb\_run}{BW_{cn\_to\_bb}} + rt\label{Equ:FinRun} \\
        & TS_{f\_loadout} = TS_{s\_stageout} + \frac{bb\_out}{BW_{cn\_to\_bb}}\label{Equ:FinMemOut} \\
        & TS_{f\_stageout} = TS_{f\_loadout} + \frac{bb\_out}{BW_{bb\_to\_io}} \label{Equ:FinOut}
\end{align}
where $TS_{f\_x}$ stands for the timestamps of finishing phase $x$,
$TS_{s\_x}$ stands for the timestamps of starting phase $x$,
$BW_{x\_to\_y}$ stands for the bandwidth between $x$ and $y$.

Though target on burst buffer system, BBSim also supports simulating non-BB HPC system.
Besides, it is not coupled with Cerberus but easy to simulating many other scheduler.
Both can be proved by the following experiments for various schedulers.

\begin{figure}[!t]
\centering
        \includegraphics[width=3.8in]{3PhaseJobFSM}
        \caption{Finite State Machine of Scheduling Simulation}
\label{Fig:JobFSM}
\end{figure}

